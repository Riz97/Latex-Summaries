\documentclass{article}
\usepackage[utf8]{inputenc}
\usepackage{xcolor}
\usepackage{graphicx}
\usepackage{subcaption}
\usepackage{amsfonts}
\usepackage{wrapfig}


\title{Geometric  Modeling}
\author{Riccardo Caprile}
\date{April 2023}

\begin{document}

\maketitle

\section{Shape Representation}

The shape representation is reached by acquiring real-world objects ( discrete sampling , points and meshes) , modeling by hand or procedural modeling (algorithms , grammars).


\subsection{Points}

Points = unordered set of 3-tuples.

Efficient point processing and modeling requires a spatial partitioning data structure to figure out neighborhoods. 

\textbf{Query} : Used to determine whether or not a given point p is inside or outside of the solid bounded by a surface S. Another typical query is the computation of point's distance to a surface.

\subsection{Parametric Curves and Surfaces}

\subsubsection{Parametric Representation}

Range of function f : X \textrightarrow{Y} , Y , X $\subseteq$ $R^m$ ,  Y $\subseteq$ $R^n$

X is a line from 0 to 1 and t is the parameter varying on this line. 

Y is the physical domain on which we describe the line.

\textbf{Planar curve} : m = 1 , n = 2 \textrightarrow{s(t) = x(t),y(t)}.

Planar curves are curves than lay on a surface , 2D curves.

\textbf{Space curve} : m = 1 , n = 3 \textrightarrow{s(t) = x(t),y(t),z(t))}

Surface in 3D : m = 2 , n = 3 

s(u,v) = (x(u,v),y(u,v),z(u,v)). As output we get a solution in $R^3$

\vspace{30mm}

\subsubsection{Parametric Curves}

Explicit curve/circle in 2D.

p : R \textrightarrow{$R^2$}

t \textrightarrow{p(t) = (x(t),y(t))}

p(t) = r(cos(t),sin(t))

The equation of the circle $x^2 + y^2 = r^2$ is an implicit version.

\begin{figure}[ht!]
  \centering
  \begin{subfigure}[b]{0.3\linewidth}
    \includegraphics[width=1\linewidth]{1.PNG}
  \end{subfigure}
\end{figure}

\textbf{Parametric Surfaces }

Sphere in 3D ,  s : $R^2$ \textrightarrow{$R^3$}

s(u,v) = r(cos(u) cos(v), sin(u) cos(v),sin(v))

$(u,v) \in [0,2\pi) x [-\pi / 2 , \pi/ 2]$



\subsubsection{Tangents and Normal}

\begin{figure}[ht!]
  \centering
  \begin{subfigure}[b]{0.8\linewidth}
    \includegraphics[width=1\linewidth]{2.PNG}
  \end{subfigure}
\end{figure}

Parametric Curves and Surfaces are easy to generate points on the curve/surface. 

But it is hard to tell if a point is on the curve/surface

\vspace{20mm}

\subsection{Implicit Curves and Surfaces}

The basic concept of implicit representations for geometric models is to characterize the whole embedding space of an object by classifying each 3D point to lie either inside , outside , or exactly on the surface S that bounds a solid object.
An implicit surface does not have any holes as long as the defining function F is continuous.

\begin{figure}[ht!]
  \centering
  \begin{subfigure}[b]{0.3\linewidth}
    \includegraphics[width=1\linewidth]{3.PNG}
  \end{subfigure}
\end{figure}

Kernel of a scalar function f : $R^m$\textrightarrow{R}.

\begin{itemize}
    \item Curve in 2D : S = ($x \in R^2 | f(x) = 0$)
    \item surface in 3D : S = (${x \in R^3 | f(x) = 0}$)
\end{itemize}

Space Partitioning :

\begin{itemize}
    \item Outside : ($x \in R^m | f(x) > 0$)
    \item Curve/Surface : ($x \in R^m | f(x) > 0$)
    \item Inside : ($x \in R^m | f(x) > 0$)
\end{itemize}

The implicit formula for circle and sphere is : $f(x,y) = x^2 + x^2 - r^2 $ and $f(x,y,z) = x^2 + y^2 + z^2 - r^2$

The normal direction to the surface (curve) is given by the gradient of the implicit function.

\begin{figure}[ht!]
  \centering
  \begin{subfigure}[b]{0.7\linewidth}
    \includegraphics[width=1\linewidth]{4.PNG}
  \end{subfigure}
\end{figure}

$\partial f$ / $\partial x$ , it represents the partial derivative of the function using x.

\section{Geometric Meshes}

\subsection{Polygonal Meshes}

Polygonal meshes are boundary representations of objects.

Faces can  be triangles or quadrilateral or mixed.

Meshes can be used as an approximation of smooth surfaces.

\begin{figure}[ht!]
  \centering
  \begin{subfigure}[b]{0.7\linewidth}
    \includegraphics[width=1\linewidth]{5.PNG}
  \end{subfigure}
\end{figure}

We are  trying to approximate the surface of the red circles using , at the beginning , a triangle with three edges. Then we double the number of edges and we can see that the error decreases.

Polygonal meshes are a good representation of approximation , arbitrary topology and piecewise smooth surfaces.


\subsubsection{Polygon}

Vertices : $v_0,v_1,...,v_n-1$

Edges : \{ $(v_0,v_1),....(V_n-1,v_0)$\}

All polygons are closed and all vertices are on a plane.

\subsubsection{Polygon Mesh}

\begin{figure}[ht!]
  \centering
  \begin{subfigure}[b]{0.2 \linewidth}
    \includegraphics[width=1\linewidth]{6.PNG}
  \end{subfigure}
\end{figure}

A finite set M of closed , simple polygons $Q_i$

The intersection of two polygons in M is either empty , or a vertex , or an edge .

M = (V,E,F) , V stands for vertices , E for edges and F for faces.

Every edge belongs to at least one polygon and each $Q_i$ defines a face of the polygonal mesh.


Vertex \textbf{degree} = number of incident edges.

\textbf{Boundary} : is the set of all edges that belong to only one polygon. If empty , then the polygonal mesh is said to be closed or watertight.

\subsubsection{Triangle Meshes }

\begin{figure}[ht!]
  \centering
  \begin{subfigure}[b]{0.7\linewidth}
    \includegraphics[width=1\linewidth]{7.PNG}
  \end{subfigure}
\end{figure}

\subsubsection{Manifolds}

A surface is a closed 2-manifold if it is everywhere locally homeomorphic (deformazione senza strappi,da un punto di vista topologico sono due oggetti uguali) to a disk.


\begin{figure}[ht!]
  \centering
  \begin{subfigure}[b]{0.3\linewidth}
    \includegraphics[width=1\linewidth]{8.PNG}
  \end{subfigure}
\end{figure}


In a manifold mesh , there are at most 2 faces sharing an edge ( boundary edges have one incident face , inner edges have two incident faces).

A manifold vertex has 1 connected ring of faces around it, or 1 connected half-ring (boundary)

\begin{figure}[ht!]
  \centering
  \begin{subfigure}[b]{0.7\linewidth}
    \includegraphics[width=1\linewidth]{9.PNG}
  \end{subfigure}
\end{figure}


\subsubsection{Polyhedron}

If closed and not intersecting , a manifold divides the space into \textbf{interior} and \textbf{exterior}.

The manifold is the boundary of the object that occupies the interior.

A closed manifold polygonal mesh is called \textbf{polyhedron}.

\subsubsection{Orientation}

Every face of a polygonal mesh is orientable. 

Clockwise vs counterclockwise order of face vertices.

Defines sign/direction of the surface normal.

\begin{figure}[ht!]
  \centering
  \begin{subfigure}[b]{0.3\linewidth}
    \includegraphics[width=1\linewidth]{10.PNG}
  \end{subfigure}
\end{figure}

A polygonal mesh is orientable , if the incident faces to every edge can be consistently oriented, if the faces are consistently oriented for every edge , the mesh is oriented.


\subsubsection{Global Topology of Meshes}

\textbf{Genus} : 1/2 the maximal number of independent closed paths that do not disconnect the graph. Informally , the number of handles.


\subsubsection{Euler-Poincarè Formula}

\textbf{Topology} : study the properties of the figure , in general of the mathematical objects , that do not change when a deformation without "strappi" , "sovrapposizioni" happen.

Theorem : the sum $\chi(M) = v - e + f$ is constant for a given surface topology , no matter which mesh we choose to cover M

For orientable meshes : $v - e + f = 2(c-g) - b = \chi(M)$

\textbf{c} is the number of connected components , \textbf{g} the genus and \textbf{b} the number of boundary loops.

\subsubsection{Implication for Mesh Storage}

Let's count the edges and faces in a closed triangle mesh : 

Ratio of edges to faces = \textbf{e = 3/2 f}, each edge belongs to exactly 2 triangles, each triangle has exactly 3 edges.

Ratio of vertices to faces: f $\sim 2v$ 2v , 2 = v - e + f = v - 3/2f + f , 2 + f/2 = v. The number of triangles is twice the number of vertices

Ratio of edges to vertices : e $\sim $ 3v. The number of edges is three times the number of vertices.

Average degree of a vertex : 6. 

\subsubsection{Triangulation}

Polygonal mesh where every face is a triangle.

It simplifies data structures , rendering , algorithms., Any polygon can be triangulated.

\subsection{Editing Operations}

\begin{itemize}
    \item Refinement operators : Producing a triangulation with more vertices/edges/faces
    \item Simplification operators : Producing a triangulation with less vertices/edges/faces
\end{itemize}

\subsubsection{Refinement Operators}

\textbf{Triangle Split}

Insert a new vertex v in a triangle t and split the triangle by connecting v to the vertices of t.

\begin{figure}[ht!]
  \centering
  \begin{subfigure}[b]{0.4\linewidth}
    \includegraphics[width=1\linewidth]{11.PNG}
  \end{subfigure}
\end{figure}

\textbf{Edge Split}

Insert a new vertex v on an edge e and split the triangles incident at e connecting v to the vertices opposite to e.

\begin{figure}[ht!]
  \centering
  \begin{subfigure}[b]{0.4\linewidth}
    \includegraphics[width=1\linewidth]{12.PNG}
  \end{subfigure}
\end{figure}

\textbf{Vertex Split}

Take a vertex v and two of its incident edges e1 and e2.

Cut along e1 and e2.

Duplicate v,e1 and e2 each in two copies.

Displace either one or both copies of v.

Connect the two copies of v with a new edge e , generating two new faces t1 and t2.

\begin{figure}[ht!]
  \centering
  \begin{subfigure}[b]{0.5\linewidth}
    \includegraphics[width=1\linewidth]{13.PNG}
    \caption{Step 1}
  \end{subfigure}
     \begin{subfigure}[b]{0.49\textwidth}
         \centering
         \includegraphics[width=1\textwidth]{14.PNG}
         \caption{Step 2}
     \end{subfigure}
\end{figure}

\vspace{30mm}

\subsubsection{Simplification Operators}

\textbf{Edge collapse(inverse of vertex split)}

Take an edge e and collapse it to a single point

The two faces incident at e also collapse to edges.

\begin{figure}[ht!]
  \centering
  \begin{subfigure}[b]{0.4\linewidth}
    \includegraphics[width=1\linewidth]{15.PNG}
  \end{subfigure}
\end{figure}

\textbf{Edge merge (inverse of edge split)}

Consider an internal vertex v with 4 incident triangle.

Delete the 4 triangles and create 2 new triangles by a new edge e that is one of the two diagonals of the quadrilateral.

\begin{figure}[ht!]
  \centering
  \begin{subfigure}[b]{0.4\linewidth}
    \includegraphics[width=1\linewidth]{16.PNG}
  \end{subfigure}
\end{figure}

\textbf{Delete Vertex (inverse of triangle split)}

Consider an internal vertex v with 3 incident triangles.

Merge the 3 triangles into a single triangle t.

\begin{figure}[ht!]
  \centering
  \begin{subfigure}[b]{0.4\linewidth}
    \includegraphics[width=1\linewidth]{17.PNG}
  \end{subfigure}
\end{figure}

\subsubsection{Neutral Editing Operator}

\textbf{Edge Swap}

Take and edge e such that its two incident faces form a convex quadrilateral Q.

Replace e with the other diagonal e' of Q

\begin{figure}[ht!]
  \centering
  \begin{subfigure}[b]{0.4\linewidth}
    \includegraphics[width=1\linewidth]{18.PNG}
  \end{subfigure}
\end{figure}

Edge swap cannot be applied on non-convex quadrilaterals.

\vspace{20mm}

\subsubsection{Boundary Cases}

\begin{figure}[ht!]
  \centering
  \begin{subfigure}[b]{0.5\linewidth}
    \includegraphics[width=1\linewidth]{19.PNG}
  \end{subfigure}
     \begin{subfigure}[b]{0.49\textwidth}
         \centering
         \includegraphics[width=1\textwidth]{20.PNG}
     \end{subfigure}
\end{figure}

\subsection{Data Structures}

What should be stored?

Geometry : 3D coordinates

Connectivity : Adjacency relationships

Attributes : Normal , color , texture coordinates. Per vertex , face , edge.

What should be supported ? 


\begin{figure}[ht!]
  \centering
  \begin{subfigure}[b]{0.3\linewidth}
    \includegraphics[width=1\linewidth]{21.PNG}
  \end{subfigure}
\end{figure}

Rendering (enumeration).

Geometry queries : What are the vertices of face 2? Is vertex A adjacent to vertex H?.

Modifications : Remove /add a vertex or a face etc.

\subsubsection{Triangle List}

The simplest way to represent a surface mesh consists of storing a set of individual polygonal faces represented by their vertex positions.

4 bytes per coordinate and since due to Euler's formula , the number of faces F is about twice the number of vertices V, this data structure consumes on average 72 bytes 


\begin{figure}[ht!]
  \centering
  \begin{subfigure}[b]{0.2\linewidth}
    \includegraphics[width=1\linewidth]{22.PNG}
  \end{subfigure}
\end{figure}

No connectivity information and it is redundant.

\subsubsection{Indexed Face Set}

Stores an array of vertices and encodes polygons as a set of indices into this array

\begin{figure}[ht!]
  \centering
  \begin{subfigure}[b]{0.3\linewidth}
    \includegraphics[width=1\linewidth]{23.PNG}
  \end{subfigure}
\end{figure}

No explicit neighborhood info.

\subsubsection{Neighborhood Relations}

\begin{figure}[ht!]
  \centering
  \begin{subfigure}[b]{0.8\linewidth}
    \includegraphics[width=1\linewidth]{24.PNG}
  \end{subfigure}
\end{figure}

\subsubsection{Half-edge structure}


Data structures for general polygon meshes are logically edge-based , since the connectivity primarily relates to the mesh edges.

Half-edge data structures split each edge into two oriented \textbf{halfedges}.

Halfedges are oriented consistently in counterclockwise order around each face and along each boundary.



For each halfedge we store a reference to : 



\begin{itemize}
    \item The vertex it points to
    \item Its adjacent face
    \item The next halfedge of the face or boundary
    \item The previous halfedge in the face
    \item Its opposite halfedge
\end{itemize}

\begin{figure}[ht!]
  \centering
  \begin{subfigure}[b]{0.3\linewidth}
    \includegraphics[width=1\linewidth]{25.PNG}
  \end{subfigure}
\end{figure}

One ring traversal : 

It start at a vertex ( in this case the one in the middle). Then we move to the outgoing halfedge , then to the twin halfedge , the next halfedge , the twin etc

\begin{figure}[ht!]
  \centering
  \begin{subfigure}[b]{0.3\linewidth}
    \includegraphics[width=1\linewidth]{26.PNG}
    \caption{Second step , outgoing halfedge}
  \end{subfigure}
\end{figure}


The problem with this implementation is that it is heavy because requires to store and manage extra pointers.

\vspace{20mm}

\section{Normal Estimation}


Assign a normal vector n at each point cloud x.

\begin{itemize}
    \item Estimate the direction by fitting a local plane (in 2D is a line , like in the case below)
    \item Find consistent global orientation by propagation. We have to check the neighbour and if the dot product between the visited node and the unvisited one is negative , the univisited will be flipped.
\end{itemize}

\begin{figure}[!htb]
\minipage{0.32\textwidth}
  \includegraphics[width=\linewidth]{27.PNG}
  
\endminipage\hfill
\minipage{0.32\textwidth}
  \includegraphics[width=\linewidth]{28.PNG}
  
\endminipage\hfill
\minipage{0.32\textwidth}%
  \includegraphics[width=\linewidth]{29.PNG}
  
\endminipage
\end{figure}

For each point x in the cloud , pick k nearest neighbors or all points in r-ball.

What we need is a plane that minimizes the sum of square distances : $min \sum dist (x_i , \Pi)^2$

And this can be done by using linear least squares.

\begin{figure}[ht!]
  \centering
  \begin{subfigure}[b]{0.5\linewidth}
    \includegraphics[width=1\linewidth]{30.PNG}
  \end{subfigure}
     \begin{subfigure}[b]{0.49\textwidth}
         \centering
         \includegraphics[width=1\textwidth]{31.PNG}
         \caption{In this case we have the orthogonal distance between the line and points}
     \end{subfigure}
\end{figure}

\subsection{Principal Component Analysis (PCA)}

PCA finds an orthogonal basis that the best represents a given dataset.

PCA finds the best approximating line/plane/orientation in terms of square distance.

\vspace{20mm}

\subsubsection{Notations}

\textbf{Input points} : $x_1,x_2,...,x_n \in R^d$

\begin{figure}[ht!]
  \centering
  \begin{subfigure}[b]{0.3\linewidth}
    \includegraphics[width=1\linewidth]{32.PNG}
  \end{subfigure}
\end{figure}

Looking for a plane passing through c with normal n

\textbf{Centroid}: $m = 1/n \sum_{i=1}^n x_i$

\textbf{Vectors from the centroid} : $y_i = x_i - m$ , a point which is in average position between all the points considered



\begin{figure}[ht!]
  \centering
  \begin{subfigure}[b]{0.8\linewidth}
    \includegraphics[width=1\linewidth]{33.PNG}
  \end{subfigure}
\end{figure}

To find the best fitting plane we have to :

\begin{itemize}
    \item Input points
    \item Compute the centroid , which is the plane origin
    \item Compute the scatter matrix S = $YY^T$ \textrightarrow{y = ($y_1 y_2 ... y_n$)} , $y_i = x_i - m$
\end{itemize}

The plane normal n is the eigenvector of S with the smallest eigenvalue.

Si dice che uno scalare $\lambda_0$ é un autovalore\textbf{(eigenvalue)} della matrice quadrata A se esiste un vettore colonna non nullo v \textbf{(eigen vector)} tale che $Av = \lambda_0v$

\vspace{50mm}

The Scatter matrix measures the variance of our data point along the direction v.

\begin{figure}[ht!]
  \centering
  \begin{subfigure}[b]{0.8\linewidth}
    \includegraphics[width=1\linewidth]{34.PNG}
  \end{subfigure}
\end{figure}



Eigenvectors of S that correspond to big eigenvalues are the directions in which the data has strong components (= large variance)

If the eigenvalues are more or less the same , there is no preferable direction.

\begin{figure}[ht!]
  \centering
  \begin{subfigure}[b]{0.8\linewidth}
    \includegraphics[width=1\linewidth]{35.PNG}
  \end{subfigure}
\end{figure}


\section{Surface Reconstruction}

How to get a mesh out of a cloud of points.

The first step is the Scanning , which results in range images.

Then Registration that bring all range images to one coordinate system 

In the end , Reconstruction that is the integration of scans into a single mesh

\textbf{Explicit Reconstruction}

\begin{enumerate}
    \item Each range scan is meshed independently 
    \item Stitch the range scans together
\end{enumerate}

Connect sample points by triangles.

Bad for noisy or misaligned data and can lead to holes or non manifold situations.

\begin{figure}[ht!]
  \centering
  \begin{subfigure}[b]{0.5\linewidth}
    \includegraphics[width=1\linewidth]{36.PNG}
    \caption{Explicit Reconstruction}
  \end{subfigure}
     \begin{subfigure}[b]{0.49\textwidth}
         \centering
         \includegraphics[width=1\textwidth]{37.PNG}
         \caption{Implicit Reconstruction}
     \end{subfigure}
\end{figure}


\textbf{Implicit Reconstruction}

\begin{enumerate}
    \item Estimate a signed distance function (SDF)
    \item Extract 0-level set mesh
\end{enumerate}

Approximation of input points. Watertight manifold by construction.


Implicit function approach, with a value $>$ 0 outside the shape and $<$ 0 inside. The zero set is the boundary where f(x) = 0

\subsection{SDF from Points and Normals}

Compute signed distance to the tangent plane of the closest point.

Normals help to distinguish between inside and outside.

\begin{figure}[ht!]
  \centering
  \begin{subfigure}[b]{0.4\linewidth}
    \includegraphics[width=1\linewidth]{38.PNG}
  \end{subfigure}
\end{figure}

\vspace{30mm}

Compute signed distance to the tangent plane of the closest point.


\begin{figure}[ht!]
  \centering
  \begin{subfigure}[b]{0.39\linewidth}
    \includegraphics[width=1\linewidth]{39.PNG}
  \end{subfigure}
     \begin{subfigure}[b]{0.39\textwidth}
         \centering
         \includegraphics[width=1\textwidth]{40.PNG}
         \caption{The function will be discontinuous}
     \end{subfigure}
\end{figure}


\subsubsection{Smooth SDF}

Instead find a smooth formulation for F.

Scattered data interpolation : F($p_i$) = 0 , F is smooth.

\begin{figure}[ht!]
  \centering
  \begin{subfigure}[b]{0.39\linewidth}
    \includegraphics[width=1\linewidth]{41.PNG}
    \caption{}
  \end{subfigure}
     \begin{subfigure}[b]{0.39\textwidth}
         \centering
         \includegraphics[width=1\textwidth]{42.PNG}
         \caption{Add off-surface constraints}
     \end{subfigure}
\end{figure}

\subsubsection{Radial Basis Function Interpolation}

\textbf{RBF} : Weighted sum of shifted , smooth kernels

$F(x) = \sum_{i = 0}^{N-1} \omega_i \phi(||x-c_i||)$

Where $\phi$ is the basis function corresponding to the center $c_i$

To find and RBF function that interpolates the displacement constraints , we use as many RBF kernels as we have constraints and place them on the constraints. 

$\{c_{3i},c_{3i+1},c_{3i+2}\} = \{p_i, p_i + \epsilon n_i,p_i - \epsilon n_i\}$

The weights $w_i$ are then found as the solution of the symmetric linear system.

\begin{figure}[ht!]
  \centering
  \begin{subfigure}[b]{0.7\linewidth}
    \includegraphics[width=1\linewidth]{43.PNG}
  \end{subfigure}
\end{figure}

Once the weight has been computed , the RBF function d has been fit to the constraints.

RBF is a global definition. A global optimization of the weights , even if the basis function are local.


\subsection{Moving Least Squares (MLS)}

Do purely \textbf{local} approximation of the SDF.

Weights change depending on where we are evaluating.

The beauty: the stitching of all local approximations, seen as one function F(x) , is smooth everywhere! We get a globally smooth function but only do local computation.

\begin{figure}[ht!]
  \centering
  \begin{subfigure}[b]{0.39\linewidth}
    \includegraphics[width=1\linewidth]{44.PNG}
    \caption{}
  \end{subfigure}
     \begin{subfigure}[b]{0.39\textwidth}
         \centering
         \includegraphics[width=1\textwidth]{45.PNG}
     \end{subfigure}
\end{figure}

\subsubsection{MOVING Least-Squares Approximation}

Polynomial least-squares approximation

\begin{figure}[ht!]
  \centering
  \begin{subfigure}[b]{0.7\linewidth}
    \includegraphics[width=1\linewidth]{46.PNG}
  \end{subfigure}
\end{figure}

\vspace{30mm}

\subsubsection{Dependence on Weight Function}

\begin{figure}[ht!]
  \centering
  \begin{subfigure}[b]{0.7\linewidth}
    \includegraphics[width=1\linewidth]{47.PNG}
  \end{subfigure}
\end{figure}


\subsection{Tessellation}

We want to approximate an implicit surface with a mesh.

Can't explicitly compute all the roots.

Solution: find an approximate roots by trapping the implicit surface in a grid.

\begin{figure}[ht!]
  \centering
  \begin{subfigure}[b]{0.7\linewidth}
    \includegraphics[width=1\linewidth]{48.PNG}
  \end{subfigure}
\end{figure}

\subsubsection{Marching Squares}

The marching squares algorithm aims at drawing lines between interpolated values along the edges of a square.

In order to display figure we want , we can consider individually each square of the grid using 16 different configuration which allows the representation of all kinds of lines in 2D space.

\begin{figure}[ht!]
  \centering
  \begin{subfigure}[b]{0.4\linewidth}
    \includegraphics[width=1\linewidth]{49.PNG}
    \caption{4 equivalence classes}
  \end{subfigure}
\end{figure}

Case 4 is ambiguous , always pick consistently to avoid problems with the resulting mesh.

\vspace{30mm}

\subsection{Marching Cubes}

Following the marching squares algorithm we can adapt our approach to the 3D case. In a 3D space we enumerate 256 different situations for the marching cubes representation

\begin{enumerate}
    \item Load 4 layers of the grid into memory
    \item Create a cube whose vertices lie on the two middle layers
    \item Classify the vertices of the cube according to the implicit function (outside,inside or on the surface)

    \begin{figure}[ht!]
  \centering
  \begin{subfigure}[b]{0.4\linewidth}
    \includegraphics[width=1\linewidth]{50.PNG}
  \end{subfigure}
\end{figure}

\item Compute case index. We have $2^8 = 256 $ cases (0/1 for each of the eight vertices). If the vertices is inside the bit relative to the vertex is set to 1 , 0 otherwise. When all the vertices are checked we got the configuration wanted expressed as a 8 bit number

    \begin{figure}[ht!]
  \centering
  \begin{subfigure}[b]{0.8\linewidth}
    \includegraphics[width=1\linewidth]{51.PNG}
  \end{subfigure}
\end{figure}

\item Using the case index , retrieve the connectivity in the look up table. In this case cut edges $e_1,e_4,e_5,e_6,e_9 and e_10$

\item Then we have to compute the position of the cut vertices by linear interpolation
\item Move to the next cube

\end{enumerate}

As before we have to make consistent choices for neighboring cubes , otherwise we get holes.

We have problems with short triangle edges.

Triangles with short edges waste resources but do not contribute to the surface mesh representation.

\subsubsection{Grid Snapping}

Solution : threshold the distances between the created vertices and the cube corners.

When the distance is smaller than $d_snap$ we snap the vertex to the cube corner.

If more than one vertex of a triangle is snapped to the same point , we discard that triangle altogether.

\section{Curves}

\subsection{Differential Geometry Basics}

Geometry of manifolds.

Things can be discovered by local observations : point + neighborhood.

    \begin{figure}[ht!]
  \centering
  \begin{subfigure}[b]{0.5\linewidth}
    \includegraphics[width=1\linewidth]{52.PNG}
  \end{subfigure}
\end{figure}

If it is a sufficiently smooth mapping can be constructed , we can look at its first and second derivatives to find : tangent , normals , curvatures etc.


\vspace{50mm}

\subsection{Curves}

We need mathematical concepts to characterize the desired curve properties.

Notions from curve geometry help with designing user interfaces for curve creation and editing.

\subsubsection{2D Parametric Curve}

    \begin{figure}[ht!]
  \centering
  \begin{subfigure}[b]{0.7\linewidth}
    \includegraphics[width=1\linewidth]{53.PNG}
    \caption{A curve is defined as \textbf{the image function of a function x}}
  \end{subfigure}
\end{figure}

The coordinates x and y are assumed to be differentiable functions (continuous, gradient exists) of t.

The same curve can be obtained with a different parametrization. For example $P_1(t) = (u,u)^T and P_2 = (u^2,u^2)^T$ describe the same curve for $t \in [0,1]$, but their parametrization is different. This illustrates that we can change the parametrization without changing the shape of the curve.

The tangent vector to the curve is at point p(t) is defined as the first derivative of the coordinate function.

The \textbf{tangent vector} corresponds to the velocity vector a time t, norm of the derivatives.


\subsubsection{Arc Length}

How long is the curve between $t_0$ and $t$? 

This can be computed as the integral of the tangent vector.


    \begin{figure}[ht!]
  \centering
  \begin{subfigure}[b]{0.7\linewidth}
    \includegraphics[width=1\linewidth]{54.PNG}
  \end{subfigure}
\end{figure}

\vspace{30mm}

\subsection{Tangent : Geometric Construction}

\textbf{Secant} : a line through two points on the curve

\textbf{Tangent}: the limiting secant as two points come together

    \begin{figure}[ht!]
  \centering
  \begin{subfigure}[b]{0.4\linewidth}
    \includegraphics[width=1\linewidth]{55.PNG}
  \end{subfigure}
\end{figure}

\subsection{Curvature}

How much does the curve turn per unit S?

    \begin{figure}[ht!]
  \centering
  \begin{subfigure}[b]{0.5\linewidth}
    \includegraphics[width=1\linewidth]{56.PNG}
  \end{subfigure}
\end{figure}

k(s) , in the book , is $||x''(s)||$ , this formula define the curvature at a point x(s).

Curvature measures how strongly a curve deviates from a straight line.

\subsubsection{Curvature: Geometric Construction}

We need to find the "best fitting circle".

    \begin{figure}[ht!]
  \centering
  \begin{subfigure}[b]{0.4\linewidth}
    \includegraphics[width=1\linewidth]{57.PNG}
  \end{subfigure}
\end{figure}

\begin{figure}[!htb]
\minipage{0.32\textwidth}
  \includegraphics[width=\linewidth]{58.PNG}
  \caption{Consider the circle passing through three points on the curve}
  
\endminipage\hfill
\minipage{0.32\textwidth}
  \includegraphics[width=\linewidth]{59.PNG}
  \caption{The limiting circles as Q and R move to P : \textbf{Osculating Circle}}
  
\endminipage\hfill
\minipage{0.32\textwidth}%
  \includegraphics[width=\linewidth]{60.PNG}
  \caption{Take the radius of the osculating circle}
  
\endminipage
\end{figure}

\vspace{30mm}

\subsubsection{Curvature of a circle}

    \begin{figure}[ht!]
  \centering
  \begin{subfigure}[b]{0.8\linewidth}
    \includegraphics[width=1\linewidth]{61.PNG}
  \end{subfigure}
\end{figure}

\subsubsection{Curvature Normal}

The curve normal vector can be defined using the relation x'' = k(s)n(s). 

\vspace{50mm}

\section{Discrete Differential Geometry of Curves}

\subsection{Discrete Planar Curves}

Piecewise linear curves , not smooth at vertices and can't take derivatives.

Generalize notions from the smooth world for the discrete case.

    \begin{figure}[ht!]
  \centering
  \begin{subfigure}[b]{0.4\linewidth}
    \includegraphics[width=1\linewidth]{62.PNG}
  \end{subfigure}
\end{figure}

\subsubsection{Tangents , Normals}


For any point on the edge , the tangent is simply the unit vector along the edge and the normal is the perpendicular vector.

\begin{figure}[!htb]
\minipage{0.32\textwidth}
  \includegraphics[width=\linewidth]{63.PNG}
  \caption{In gray Normal , in blue the Tangent}
  
\endminipage\hfill
\minipage{0.22\textwidth}
  \includegraphics[width=\linewidth]{64.PNG}
  \caption{For vertices , we have many options}
  
\endminipage\hfill
\minipage{0.42\textwidth}%
  \includegraphics[width=\linewidth]{65.PNG}
  \caption{Can choose to average the adjacent edge normals for obtaining the normal of the vertex v}
  
\endminipage
\end{figure}


    \begin{figure}[ht!]
  \centering
  \begin{subfigure}[b]{0.6\linewidth}
    \includegraphics[width=1\linewidth]{66.PNG}
    \caption{Or we can weight by edge lengths}
  \end{subfigure}
\end{figure}

\subsubsection{Inscribed Polygon , p}

Connection between discrete and smooth.

Finite number of vertices each lying on the curve, connected by straight edges.


    \begin{figure}[ht!]
  \centering
  \begin{subfigure}[b]{0.3\linewidth}
    \includegraphics[width=1\linewidth]{67.PNG}
  \end{subfigure}
\end{figure}

\subsubsection{The length of a Discrete Curve}

    \begin{figure}[ht!]
  \centering
  \begin{subfigure}[b]{0.6\linewidth}
    \includegraphics[width=1\linewidth]{68.PNG}
  \end{subfigure}
\end{figure}

\subsubsection{The Length of a Continuous Curve}

    \begin{figure}[ht!]
  \centering
  \begin{subfigure}[b]{0.6\linewidth}
    \includegraphics[width=1\linewidth]{69.PNG}
  \end{subfigure}
\end{figure}

\vspace{50mm}

\subsubsection{Curvature of a Discrete Curve}

Curvature is the change in normal direction as we travel along the curve.


\begin{figure}[!htb]
\minipage{0.32\textwidth}
  \includegraphics[width=\linewidth]{70.PNG}
  \caption{No change along each edge, curvature is zero along edges}
  
\endminipage\hfill
\minipage{0.22\textwidth}
  \includegraphics[width=\linewidth]{71.PNG}
  \caption{Normal changes at vertices , record the turning angle!}
  
\endminipage\hfill
\minipage{0.42\textwidth}%
  \includegraphics[width=\linewidth]{72.PNG}
  \caption{Same as the turning angle between the edges}
  
\endminipage
\end{figure}

Zero along the edges.

Turning angle at the vertices = the change in normal direction

    \begin{figure}[ht!]
  \centering
  \begin{subfigure}[b]{0.6\linewidth}
    \includegraphics[width=1\linewidth]{73.PNG}
  \end{subfigure}
\end{figure}

$tsc(p) = \sum_{i=1}^n a_i$ , sum of turning angles

\subsubsection{Discrete Gauss Map}


\begin{figure}[ht!]
  \centering
  \begin{subfigure}[b]{0.39\linewidth}
    \includegraphics[width=1\linewidth]{74.PNG}
    \caption{Edges map to points , vertices map to arcs}
  \end{subfigure}
     \begin{subfigure}[b]{0.39\textwidth}
         \centering
         \includegraphics[width=1\textwidth]{75.PNG}
         \caption{Turning number well-defined for discrete curves}
     \end{subfigure}
\end{figure}

\vspace{35mm}

\section{Surfaces}

\subsection{Parametric Representation of surfaces}

\begin{itemize}
    \item Surface : S $\subset R^3$
    \item Parameter Domain : $\Omega \subset R^2$
    \item Mapping: $f: \Omega$ \textrightarrow{S}
\end{itemize}

f is a bi-variate vector function mapping points of the 2D plane to points in 3D space

We can define an injective parametrization iff surface S is homeomorphic to a (punctured) disk.


    \begin{figure}[ht!]
  \centering
  \begin{subfigure}[b]{0.4\linewidth}
    \includegraphics[width=1\linewidth]{76.PNG}
  \end{subfigure}
\end{figure}

For a surface S that is no homeomorphic to a punctured disk , parametrization needs to cut S at the image of borders of $\Omega$

We need to find a way to unfold the surface of the world, in order to obtain a flat 2D surface.


    \begin{figure}[ht!]
  \centering
  \begin{subfigure}[b]{0.4\linewidth}
    \includegraphics[width=1\linewidth]{77.PNG}
  \end{subfigure}
\end{figure}


\vspace{60mm}

\subsection{Tangent Plane}

f(u,v) = $(x(u,v),y(u,v),z(u,v))^T$

\begin{itemize}
    \item Jacobian of f : $J_f = [f_u.f_v]$, derivate parziali di u e v
    \item Tangent plane at p : $T_p = \{p + af_u + bf_v : a,b \in R\}$
    \item Taylor expansion of f : $f(y) = f(x) + J_f(y-x) + ...$
\end{itemize}


    \begin{figure}[ht!]
  \centering
  \begin{subfigure}[b]{0.2\linewidth}
    \includegraphics[width=1\linewidth]{78.PNG}
  \end{subfigure}
\end{figure}


\subsection{Surface Normal}

Surface normal at a point p is the unit-length normal vector n of tangent plane $T_p$ pointing outwards from S.

    \begin{figure}[ht!]
  \centering
  \begin{subfigure}[b]{0.6\linewidth}
    \includegraphics[width=1\linewidth]{79.PNG}
  \end{subfigure}
\end{figure}

\vspace{30mm}

\subsection{Directional derivatives}

Given a direction vector $\bar w$ = $(u_w,v_w)^T$

Straight line in parameter space (u,v) = $(u_x,v_x) + t\bar w$

Corresponds to a curve on S : $C_w(t) = f(u_x + tu_w, v_x + tv_w)$

    \begin{figure}[ht!]
  \centering
  \begin{subfigure}[b]{0.3\linewidth}
    \includegraphics[width=1\linewidth]{80.PNG}
  \end{subfigure}
\end{figure}

Directional derivative w of f at $(u_x,v_x)$ wrt $\bar w$ is the tangent to $C_w (t)$ computed at t=0

By the chian rule w = $J_f\bar w$


\subsection{First Fundamental Form}

Basically the Jacobian matrix correspond to the linear map that transforms a vector $\bar w$ in parameter space into a tangent vector w on the surface


    \begin{figure}[ht!]
  \centering
  \begin{subfigure}[b]{0.5\linewidth}
    \includegraphics[width=1\linewidth]{81.PNG}
  \end{subfigure}
\end{figure}

The first fundamental form I defines an inner product on the tangent space of S.

Beside measuring angles , we can use this inner product to determine the squared length of a tangent vector w as $||w||^2 = \bar w^TI\bar w$

\subsection{Measuring Areas}

The first fundamental form can be used to measure areas on the surface : Area of a region f(U) where U is a region in $\Omega$:

    \begin{figure}[ht!]
  \centering
  \begin{subfigure}[b]{0.5\linewidth}
    \includegraphics[width=1\linewidth]{82.PNG}
  \end{subfigure}
\end{figure}

\subsection{Gauss Map}

The Gauss map sends a point on the surface to the outward pointing unit normal vector :

    \begin{figure}[ht!]
  \centering
  \begin{subfigure}[b]{0.5\linewidth}
    \includegraphics[width=1\linewidth]{83.PNG}
  \end{subfigure}
\end{figure}

\subsection{Normal Curvature}

To extend the notion of curvature from curves to surfaces, we look at the curvature of curves embedded in the surface.

Let $t = u_tx_u + v_tx_v$ be a tangent vector at a surface point p represented as $\bar t$ = $(u_t,v_t)$ in parameter space.

The normal curvature $k_n(\bar t) $ at p is the curvature of the planar curve created by intersecting the surface at p with the plane spanned by t and the surface normal n

    \begin{figure}[ht!]
  \centering
  \begin{subfigure}[b]{0.5\linewidth}
    \includegraphics[width=1\linewidth]{84.PNG}
    \caption{t is the tangent vector}
  \end{subfigure}
\end{figure}


We can choose a specific curve c that has the normal aligned to n.

The curvature of c curve at p is the normal curvature wrt t.

We can express normal curvature in direction $\bar t$ as :

    \begin{figure}[ht!]
  \centering
  \begin{subfigure}[b]{0.5\linewidth}
    \includegraphics[width=1\linewidth]{85.PNG}
    \caption{t is the tangent vector}
  \end{subfigure}
\end{figure}

Where II denotes the \textbf{second fundamental form} defined as :

    \begin{figure}[ht!]
  \centering
  \begin{subfigure}[b]{0.5\linewidth}
    \includegraphics[width=1\linewidth]{86.PNG}
  \end{subfigure}
\end{figure}

\subsection{Principal Curvatures}

Consider all possible directions t on the tangent plane at p.

There are exactly two directions :

\begin{itemize}
    \item $t_1$ such that $k_1 = k(t_1)$ is maximum
     \item $t_1$ such that $k_2 = k(t_2)$ is maximum

\end{itemize}

k1 and k2 are called the principal curvatures and t1,t2 are called the principal directions of curvature.



\subsection{Euler Theorem}


    \begin{figure}[ht!]
  \centering
  \begin{subfigure}[b]{0.\linewidth}
    \includegraphics[width=1\linewidth]{87.PNG}
  \end{subfigure}
\end{figure}

\subsection{Intrinsic Geometry}

In differential geometry , properties that only depend on the first fundamental form are called intrinsic.

\subsection{Darboux Frame}

Principal directions have arbitrary orientations.

Orientations can be selected such that $(t_1,t_2,n)$ form a right-handed 3D frame with origin at p and n points outward.

The surface in a sufficiently small neighborhood of p can be expressed in explicit form wrt to the Darboux frame.

\subsection{Shape Operator}

It is a linear operator that , when applied to a tangent vector t tells how the surface normal varies when moving along t on S.

In local coordinates given by frame $(f_u,f_v)$ it is represented as a 2x2 matrix defined at each point p through the coefficients of the first and second fundamental form.

The principal directions of curvature are the eigenvector of Sh , the principal curvatures are the eigenvalues of Sh.

The shape operator is described by a diagonal matrix of principal curvatures in the frame $(t_1,t_2)$


\begin{figure}[ht!]
  \centering
  \begin{subfigure}[b]{0.49\linewidth}
    \includegraphics[width=1\linewidth]{88.PNG}
  \end{subfigure}
     \begin{subfigure}[b]{0.49\textwidth}
         \centering
         \includegraphics[width=1\textwidth]{89.PNG}
     \end{subfigure}
\end{figure}

\section{Discrete Differential Geometry of Surfaces Part 1}

\subsection{Differential Geometry on Meshes}

Assumption : Meshes are piecewise linear approximations of smooth surfaces.

Meshes have straight line edges and flat faces.

Differential properties of the approximated surface are estimated on the mesh

\subsection{Discrete Integration}

Vertex neighborhood that partition the mesh into disjoint regions are often preferred :

\begin{itemize}
    \item Barycentric Cells , connects the triangle barycenters  with the edge midpoints
    \item Voronoi Cells, replace the triangle barycenters with triangle circumcenter (il centro di una circonferenza circoscritta ad un triangolo)
    \item Mixed Voronoi Cells
\end{itemize}

    \begin{figure}[ht!]
  \centering
  \begin{subfigure}[b]{0.7\linewidth}
    \includegraphics[width=1\linewidth]{91.png}
  \end{subfigure}
\end{figure}

\subsection{Surface Normal via Discrete Integration}

Many operations require normal vectors.

Normal vectors for individual triangles t = $(v_i,v_j,v_j)$ is the normal of the plane containing t :

    \begin{figure}[ht!]
  \centering
  \begin{subfigure}[b]{0.7\linewidth}
    \includegraphics[width=1\linewidth]{90.png}
  \end{subfigure}
\end{figure}

\subsection{Area and Volume via Discrete Integration}

    \begin{figure}[ht!]
  \centering
  \begin{subfigure}[b]{0.7\linewidth}
    \includegraphics[width=1\linewidth]{92.png}
  \end{subfigure}
\end{figure}

\subsection{Discrete Curvature}

\subsubsection{Shape Operator via Surface Fitting}

The radius r of the neighborhood of each point p is used as a scale parameter.

\begin{enumerate}
    \item Gather all vertices in a local neighborhood of radius r
    \item  set a local frame at p: $(u,v,n_p)$ where $n_p$ is the surface normal at p and u,v are any two orthogonal unit vectors spanning the tangent plane.
    \item Discard all vertices $v_i$ such that $n_i \cdot w < 0$
    \item Express all vertices of the neighborhood the local frame
    \item Fit to these points a polynomial of degree two through p, $f(u,v) = au^2 + bv^2 + cuv + du + ev$
    \item Shape operator at p is computed analytically via first and second fundamental forms of f at the origin.
\end{enumerate}

Since f is a polynomial , coefficients E,F,G,L,M,N of the fundamental forms can be computed easily in closed form.

SVD decomposition of the shape operator provides principal curvatures directions.

\subsubsection{Shape Operator via normal Curvature}

\begin{enumerate}
    \item Project all edges $(v_i,v_j)$ incident at $v_i$ on the tangent plane to obtain tangent directions $w_{ij}$.
    \item For each direction $w_{ij}$ evaluate normal curvature $k_{ij}$ by local curve fitting
    \item Average results from normal curvatures to compute the shape operator
\end{enumerate}

\begin{figure}[ht!]
  \centering
  \begin{subfigure}[b]{0.49\linewidth}
    \includegraphics[width=1\linewidth]{93.PNG}
  \end{subfigure}
     \begin{subfigure}[b]{0.49\textwidth}
         \centering
         \includegraphics[width=1\textwidth]{94.PNG}
     \end{subfigure}
\end{figure}

\section{Discrete Differential Geometry of Surfaces - Functions on Surfaces}

\subsection{Gradient}

The gradient of f is defined as: $\bigtriangledown$f = $(f_u,f_v)^T$

The gradient $\bigtriangledown$f(p) of a function at a point p is a vector on the plane pointing to the direction of maximal ascent of f and having a size proportional to the rate of ascent.

f: M \textrightarrow{R} defined on a manifold , the gradient $\bigtriangledown f_M$ at a point p is a vector in the tangent plane at p pointing to the direction of maximal ascent of f and having a size proportional to the rate of ascent.

\vspace{30mm}

\subsection{Laplace Operator}

    \begin{figure}[ht!]
  \centering
  \begin{subfigure}[b]{0.9\linewidth}
    \includegraphics[width=1\linewidth]{95.png}
  \end{subfigure}
\end{figure}


\subsection{Laplace-Beltrami Operator}

Extends the concept explained before to functions defined on surface

    \begin{figure}[ht!]
  \centering
  \begin{subfigure}[b]{0.7\linewidth}
    \includegraphics[width=1\linewidth]{96.png}
  \end{subfigure}
\end{figure}

\subsubsection{Harmonic Functions and Surfaces}

A function is said to be \textbf{harmonic} if its Laplacian is null.

A \textbf{minimal surface} has null mean curvature.

Harmonic functions and minimal surfaces are especially smooth.

\vspace{30mm}

\subsection{Discrete Setting}

\subsubsection{Barycentric Coordinates}


    \begin{figure}[ht!]
  \centering
  \begin{subfigure}[b]{0.7\linewidth}
    \includegraphics[width=1\linewidth]{97.png}
  \end{subfigure}
\end{figure}


\begin{figure}[ht!]
  \centering
  \begin{subfigure}[b]{0.49\linewidth}
    \includegraphics[width=1\linewidth]{98.PNG}
  \end{subfigure}
     \begin{subfigure}[b]{0.49\textwidth}
         \centering
         \includegraphics[width=1\textwidth]{99.PNG}
     \end{subfigure}
\end{figure}

\subsubsection{Piecewise Linear Functions on Meshes}

We assume a piecewise linear function f that is given at each mesh vertex and interpolated linearly within each triangle.

    \begin{figure}[ht!]
  \centering
  \begin{subfigure}[b]{0.7\linewidth}
    \includegraphics[width=1\linewidth]{100.png}
  \end{subfigure}
\end{figure}

\vspace{20mm}

\subsubsection{The Hat Function}

    \begin{figure}[ht!]
  \centering
  \begin{subfigure}[b]{0.8\linewidth}
    \includegraphics[width=1\linewidth]{101.png}
  \end{subfigure}
\end{figure}

\subsubsection{Gradient of the Hat Function}

    \begin{figure}[ht!]
  \centering
  \begin{subfigure}[b]{0.8\linewidth}
    \includegraphics[width=1\linewidth]{102.png}
  \end{subfigure}
\end{figure}

$\bot$ denotes a counterclockwise rotation by 90 degrees in the triangle plane and A is the Area of the triangle.

\subsection{Discrete Laplace-Beltrami uniform}

Directly measures the difference between function at a vertex wrt the average of its neighbors.

    \begin{figure}[ht!]
  \centering
  \begin{subfigure}[b]{0.7\linewidth}
    \includegraphics[width=1\linewidth]{103.png}
  \end{subfigure}
\end{figure}

Simple and efficient.

One assumption is that all triangles are equilateral.

The resulting vector can be a non zero even for a planar configuration of vertices. However, in such a setting we would expect a zero Laplacian since the mean curvature over the entire mesh is zero.

This indicates that the uniform Laplacian is not an appropriate discretization for non-uniform meshes.


\subsection{Discrete Laplace-Beltrami cotangent}

We integrate the value of the Laplacian over the averaging region A(v) about a vertex v.

    \begin{figure}[ht!]
  \centering
  \begin{subfigure}[b]{0.7\linewidth}
    \includegraphics[width=1\linewidth]{104.png}
  \end{subfigure}
\end{figure}

To simplify the integrate we make use of the divergence theorem for a vector-valued function F


    \begin{figure}[ht!]
  \centering
  \begin{subfigure}[b]{0.7\linewidth}
    \includegraphics[width=1\linewidth]{105.png}
  \end{subfigure}
\end{figure}

\subsection{Laplacian Matrix}

$w_i$ : vertex weight

$w_{ij}$ : edge weights

Mass matrix: M = diag($w_1,....,w_n)$ \textbf{diagonal matrix of vertex weights}

Stifness matrix : $L_w$ defined as 

    \begin{figure}[ht!]
  \centering
  \begin{subfigure}[b]{0.7\linewidth}
    \includegraphics[width=1\linewidth]{106.png}
  \end{subfigure}
\end{figure}

is the symmetric matrix of edge weights.

\textbf{L = $M^{-1} L_w$}

Then we have :

    \begin{figure}[ht!]
  \centering
  \begin{subfigure}[b]{0.7\linewidth}
    \includegraphics[width=1\linewidth]{107.png}
  \end{subfigure}
\end{figure}

both L and $L_w$ are sparse and M is diagonal

\subsubsection{Linear system involving the Laplacian}

Suppose we have a linear system : Lx = b using the asymmetric cotangent Laplacian

The right hand side vector b represents a scalar field , storing the desired Laplacian value at each vertex.

Solving symmetric sparse linear systems matrices is much more efficient so we prefer to solve : $L_wx = Mb$

\subsection{Discrete Laplace-Beltrami Geometric meaning}




\begin{figure}[ht!]
  \centering
  \begin{subfigure}[b]{0.49\linewidth}
    \includegraphics[width=1\linewidth]{108.PNG}
  \end{subfigure}
     \begin{subfigure}[b]{0.49\textwidth}
         \centering
         \includegraphics[width=1\textwidth]{109.PNG}
     \end{subfigure}
\end{figure}




    \begin{figure}[ht!]
  \centering
  \begin{subfigure}[b]{0.7\linewidth}
    \includegraphics[width=1\linewidth]{110.png}
  \end{subfigure}
\end{figure}


\vspace{20mm}

\section{Smoothing and Fairing}

\subsection{Surface Smoothing - Motivation}

Scanned surfaces can be noisy.

Noise produces artefacts at high frequency.

Removing noise makes the surface smoother.

Marching Cubes meshes can be ugly.

\subsection{How to measure smoothness?}

 C

\subsubsection{Which Curvature}

\begin{itemize}
    \item Prinicipal curvature : Nonlinear and discountinuous operator in the definition
    \item Gauss curvature K : Intrinsic only , insensitive to embedding in 3D space
    \item Mean curvature H : Relatively simple to extract on meshes via Laplace-Beltrami
\end{itemize}

\vspace{35mm}

\subsection{Smoothing by Diffusion Flow}

\subsubsection{Example of Smoothing Curves}

\begin{figure}[ht!]
  \centering
  \begin{subfigure}[b]{0.49\linewidth}
    \includegraphics[width=1\linewidth]{112.PNG}
  \end{subfigure}
     \begin{subfigure}[b]{0.49\textwidth}
         \centering
         \includegraphics[width=1\textwidth]{113.PNG}
     \end{subfigure}
\end{figure}

Going on with more iterations we are going to smooth the mesh

\subsection{Diffusion Flow - General Scheme}

Model for the time-dependent process of smoothing a given signal.

Diffusion flow is modeled by the \textbf{diffusion equation} , describing a signal over time :

    \begin{figure}[ht!]
  \centering
  \begin{subfigure}[b]{0.3\linewidth}
    \includegraphics[width=1\linewidth]{114.png}
  \end{subfigure}
\end{figure}

A function f obeying to to the above equation becomes smoother and smoother for increasing values of t.

f(x,0) is the function at its initial state.

Parameter $\lambda$ sets the speed at which the function is smoothed.

This technique is used to blur images and smooth terrain surfaces 

The diffusion equation is a Partial Differential Equation.

We discretize it both in space and in time.

For the spatial discretization we replace the function f by its sample values at the mesh vertices f(t) = $f(v_1,t),...,f(v_n,t))^T$

For the temporal discretization we divide the time axis into regular intervals of size h, yielding time steps $\{t,t+h,t+2h..\}$, approximating the time derivative by finite differences : 

    \begin{figure}[ht!]
  \centering
  \begin{subfigure}[b]{0.4\linewidth}
    \includegraphics[width=1\linewidth]{115.png}
  \end{subfigure}
\end{figure}

\subsubsection{On meshes : smoothing as mean curvature flow}


    \begin{figure}[ht!]
  \centering
  \begin{subfigure}[b]{0.7\linewidth}
    \includegraphics[width=1\linewidth]{116.png}
  \end{subfigure}
\end{figure}

\vspace{30mm}


\subsubsection{Taubin Smoothing : Explicit Steps}

    \begin{figure}[ht!]
  \centering
  \begin{subfigure}[b]{0.7\linewidth}
    \includegraphics[width=1\linewidth]{117.png}
  \end{subfigure}
\end{figure}

$\lambda > 0$ to smooth and $\mu < 0 $ to inflate

\subsection{Smoothing as Optimization}


    \begin{figure}[ht!]
  \centering
  \begin{subfigure}[b]{0.7\linewidth}
    \includegraphics[width=1\linewidth]{118.png}
  \end{subfigure}
\end{figure}

\vspace{50mm}

\subsection{Smoothing as filtering}

\subsubsection{Fourier Analysis}

Represent a function as weighted sum of sines and cosines

\begin{figure}[!htb]
\minipage{0.32\textwidth}
  \includegraphics[width=\linewidth]{119.PNG}
  \caption{f(x) = $a_0 + a_1cos(x)$}
  
\endminipage\hfill
\minipage{0.32\textwidth}
  \includegraphics[width=\linewidth]{120.PNG}
\caption{f(x) = $a_0 + a_1cos(x) + a_2cos(3x)$}  
  
\endminipage\hfill
\minipage{0.32\textwidth}%
  \includegraphics[width=\linewidth]{121.PNG}
 \caption{f(x) = $a_0 + a_1cos(x) + a_2cos(3x) + a_3cos(5x) + a_4cos(7x) + ...$}  
  
\endminipage
\end{figure}


    \begin{figure}[ht!]
  \centering
  \begin{subfigure}[b]{0.7\linewidth}
    \includegraphics[width=1\linewidth]{122.png}
  \end{subfigure}
\end{figure}

The Fourier transform maps a function from its representation f(x) in the spatial domain to its representation $F(\omega)$ in the frequency domain.

\subsection{Extend Fourier to meshes?}

Fourier basis functions are eigenfunctions of the Laplace operator.

On meshes : take the eigenvectors of the Laplace-Beltrami matrix

Take you L-B matrix L , compute the eigenvectors with k smallest eigenvalues, and reconstruct the mesh geometry from these eigenvectors.

\subsection{Fairing}

Fairing has the purpose to compute surfaces that are as smooth as possible.

Principle of simplest shape : the surface should be free of any unnecessary details or oscillations.

\vspace{10mm}

\section{Mesh Parametrization}

\subsection{Parametrization - Definition}

The notion of parametrization attaches a geometric coordinate system to the object

Mapping P between a 2D domain $\Omega$ and the mesh S embedded in 3D.

Each mesh vertex has a corresponding 2D position : $U(v_i) = (u_i,v_i)$


    \begin{figure}[ht!]
  \centering
  \begin{subfigure}[b]{0.5\linewidth}
    \includegraphics[width=1\linewidth]{123.png}
  \end{subfigure}
\end{figure}

Parametrization allows us to do many things in 2D and then map those actions onto the 3D surface.

One goal of parametrization is Texture Mapping : put the sruface into a one-to-one correspondence with an image.

Another goal is Remeshing : the coordinate system defined by the parametrization facilitates converting from a mesh representation into an alternative one.

To summarize , a parametrization of a 3D surface is a function putting this surface in one-to-one correspondence with a 2D domain.

    \begin{figure}[ht!]
  \centering
  \begin{subfigure}[b]{0.5\linewidth}
    \includegraphics[width=1\linewidth]{124.png}
  \end{subfigure}
\end{figure}

\subsection{How to Measure Distortion?}

\subsubsection{Measures of Local Distortio}

    \begin{figure}[ht!]
  \centering
  \begin{subfigure}[b]{0.5\linewidth}
    \includegraphics[width=1\linewidth]{125.png}
  \end{subfigure}
\end{figure}

$p_u = \partial p(u,v) / \partial u$ , $p_v = \partial p(u,v) / \partial v$ 

    \begin{figure}[ht!]
  \centering
  \begin{subfigure}[b]{0.7\linewidth}
    \includegraphics[width=1\linewidth]{126.png}
  \end{subfigure}
\end{figure}

\subsubsection{Distortion on Triangle Meshes?}

Triangle in 3D is mapped to triangle in 2D.

Unique affine mapping.

SVD of the Jacobian reveals directions of extreme stretching

    \begin{figure}[ht!]
  \centering
  \begin{subfigure}[b]{0.5\linewidth}
    \includegraphics[width=1\linewidth]{127.png}
  \end{subfigure}
\end{figure}

The possible distortion measures are : $E(T) = \sqrt{\sigma_1^2 + \sigma_2^2}$ or $E(T) = max\{\sigma_1 , 1/\sigma_2\}$

\subsection{How to Compute (Good) Parametrizations? and Quickly?}

\subsubsection{Harmonic-Mapping - Idea}

Want to flatten the mesh - no curvature \textrightarrow{ Laplace operator gives zero}

Because the Laplacian measures the regularity of a function. For instance , for a linear function the Laplacian is equal to zero. Minimizing the Laplacian of u and v results in smooth parametric coordinates, in other words , this also minimizes the distortion of the parametrization.

\subsubsection{Tutte's barycentric mapping Theorem}

Given a triangulated surface homeomorphic to a disk :

\begin{enumerate}
    \item If the (u,v) coordinates at the boundary vertices lie on a convex polygon
    \item If the coordinates of the internal vertices are convex combination of their neighbors
    \item Then the (u,v) coordinates form a valid parametrization
\end{enumerate}

\subsubsection{Convex Mapping}


Boundary vertices are fixed.

Convex weights in the Laplacian matrix.

\begin{figure}[ht!]
  \centering
  \begin{subfigure}[b]{0.49\linewidth}
    \includegraphics[width=1\linewidth]{128.PNG}
  \end{subfigure}
     \begin{subfigure}[b]{0.49\textwidth}
         \centering
         \includegraphics[width=1\textwidth]{129.PNG}
     \end{subfigure}
\end{figure}

Solve the linear system Lu = 0 , the values of the boundary vertices are known and thus substituted

\subsection{Harmonic Mapping}

Inner mesh edges as springs.

Find minimum-energy state where all vertices lie in the 2D plane.

\begin{figure}[!htb]
\minipage{0.32\textwidth}
  \includegraphics[width=\linewidth]{130.PNG}
  
\endminipage\hfill
\minipage{0.32\textwidth}
  \includegraphics[width=\linewidth]{131.PNG}
  
\endminipage\hfill
\minipage{0.32\textwidth}%
  \includegraphics[width=\linewidth]{132.PNG}

  
\endminipage
\end{figure}

\subsection{Boundary-Free Parametrization}

    \begin{figure}[ht!]
  \centering
  \begin{subfigure}[b]{0.6\linewidth}
    \includegraphics[width=1\linewidth]{133.png}
  \end{subfigure}
\end{figure}

Conformality can be imposed as a constraint on the Jacobian: the two partial derivatives are orthogonal , and they have the same length.

We define the parametrization P by defining its inverse U.

We define U as a per vertex(u,v) assignment: our variables are the (u,v) positions of the vertices in $\Omega$.

The vectors $\partial P/\partial u$ and  $\partial P/\partial v$ are linear with the variables constant inside triangles.

Such vectors can be derived triangle by triangle by inverting the derivatives of U

Conformality : $\partial P/\partial u$ x n = $\partial P/\partial v$, n normal of triangle on surface is known ,  $\partial P/\partial u$ x n - $\partial P/\partial v$ = 0

\subsection{Global Parametrization}

All the algorithms that we studied until now are able to parametrize a part of a mesh homeomorphic to a disk.

To be able to parametrize arbitrary shapes we cut a mesh into parts , and we parametrize every part independently.

    \begin{figure}[ht!]
  \centering
  \begin{subfigure}[b]{0.7\linewidth}
    \includegraphics[width=1\linewidth]{134.png}
  \end{subfigure}
\end{figure}

    \begin{figure}[ht!]
  \centering
  \begin{subfigure}[b]{0.7\linewidth}
    \includegraphics[width=1\linewidth]{135.png}
  \end{subfigure}
  \caption{We have to cut the shape , in this way we can parametrixe it}
\end{figure}

\subsubsection{How to handle cuts?}

\begin{itemize}
    \item We can ignore them and parametrize every chart separately. The collection of the separate chart can be grouped in a square with a packing algorithm. Segmentation.
    \item We want to impose continuity of the derivatives of the parametization across cuts. Packing.
\end{itemize}

\section{Mesh Deformation}

Shape deformation is used for animation , editing and simulation.

The deformation of a given surface S into the desired surface S' is described by a displacement function d that associates to each point p $\in$ S a displacement vector d(p). By this map the given surface S to its deformed version S' : $S' = \{ p + d(p) | p \in S \}$

For a discrete triangle mesh the displacement function d is piecewise linear, such that it is fully defined by the displacement vectors $d_i = d(p_i)$ of the original mesh vertices.

The user controls the deformation by prescribing displacements $\bar d_i$ for a set of so-called handle points $p_i$ and by constraining parts which stay fixed during the deformation.

\subsection{Surface-Based Deformations}

A simple popular approach works by propagating the user-defined handle transformation within the deformation region. After specifying the support region R of the deformation and a handle H within it , the handle is transformed using some modelling interface.

    \begin{figure}[ht!]
  \centering
  \begin{subfigure}[b]{0.7\linewidth}
    \includegraphics[width=1\linewidth]{136.png}
  \end{subfigure}
\end{figure}

\textbf{Example of Surface-Based Deformations}


    \begin{figure}[ht!]
  \centering
  \begin{subfigure}[b]{0.4\linewidth}
    \includegraphics[width=1\linewidth]{137.png}
  \end{subfigure}
\end{figure}

Consists of three main operators : 1) the decomposition operator which separates low and high frequencies, 2)Editing operator , which deforms the low frequency components, 3)Reconstruction operator , which adds the details the details back onto the modified base surface.

Basically we need to find a mesh that optimizes some objective functional and satisfies modeling constraints : $x_def = argmin E(x')$

\subsection{Space Deformations}

Global and local deformation of solids. F : $R^3$ \textrightarrow{$R^3$}

Design a set of coordinates for all points in R with respect to the "cage" vertices.

Each point x can be represented as a weighted sum of cage points $p_i$ : x = $\sum_{i=1}^k w_i (x)p_i$, when the cage changes , the coordinate stay the same but we have to substitute the new cage geometry : x' = $\sum_{i=1}^k w_i (x)p_i$x = $\sum_{i=1}^k w_i (x)p_i'$


    \begin{figure}[ht!]
  \centering
  \begin{subfigure}[b]{0.3\linewidth}
    \includegraphics[width=1\linewidth]{138.png}
  \end{subfigure}
\end{figure}

\subsection{Surface-Based Differential Deformations}

\subsubsection{ROI-Handle Editing Metaphor}

    \begin{figure}[ht!]
  \centering
  \begin{subfigure}[b]{0.5\linewidth}
    \includegraphics[width=1\linewidth]{139.png}
  \end{subfigure}
\end{figure}

\subsection{Recap: Differential Coordinates}

Detail = smooth(surface) - surface

Smoothing = averaging.

\subsubsection{Simple Laplacian Editing}

Preserve mean curvature normal at every point in the ROI (part of surface we are interestd in deforming )

    \begin{figure}[ht!]
  \centering
  \begin{subfigure}[b]{0.5\linewidth}
    \includegraphics[width=1\linewidth]{140.png}
  \end{subfigure}
\end{figure}


\subsubsection{Simplifying the Laplacian Energy}


    \begin{figure}[ht!]
  \centering
  \begin{subfigure}[b]{0.8\linewidth}
    \includegraphics[width=1\linewidth]{141.png}
  \end{subfigure}
\end{figure}


\subsection{Fundamental Problem : Invariance to Transformations}

The basic Laplacian operator is translation-invariant , but not rotation-invariant.

E(x') attempts to preserve the original globe orientation of the details (the normal directions).


    \begin{figure}[ht!]
  \centering
  \begin{subfigure}[b]{0.7\linewidth}
    \includegraphics[width=1\linewidth]{142.png}
  \end{subfigure}
\end{figure}

We need a rigid-invariant energy : E(x') = $\sum_{i=1}^n A_i ||\bigtriangleup (x_i') - \delta_i ||^2$


\subsection{Fixing Local Rotations : Multiresolution Approach}

\begin{figure}[!htb]
\minipage{0.32\textwidth}
  \includegraphics[width=\linewidth]{143.png}
  
\endminipage\hfill
\minipage{0.32\textwidth}
  \includegraphics[width=\linewidth]{144.png}
  
\endminipage\hfill
\minipage{0.32\textwidth}%
  \includegraphics[width=\linewidth]{145.png}

  
\endminipage
\end{figure}


\begin{figure}[!htb]
\minipage{0.32\textwidth}
  \includegraphics[width=\linewidth]{146.png}
  
\endminipage\hfill
\minipage{0.32\textwidth}
  \includegraphics[width=\linewidth]{147.png}
  
\endminipage\hfill
\minipage{0.32\textwidth}%
  \includegraphics[width=\linewidth]{148.png}

  
\endminipage
\end{figure}

    \begin{figure}[ht!]
  \centering
  \begin{subfigure}[b]{0.4\linewidth}
    \includegraphics[width=1\linewidth]{149.png}
  \end{subfigure}
\end{figure}

\vspace{30mm}

\section{Splines}

General idea : smooth curve/surface in parametric form.

Defined by combining two ingredients : 

\begin{itemize}
    \item Control polygon/grid built upon a small set of \textbf{control points}
    \item \textbf{Basis functions} to compute affine averages of point positions.
\end{itemize}

The curve/surface can be edited by moving the control points or reshaping the basis functions

\vspace{100mm}

\subsection{Curves}

\subsubsection{Spline Curves}

Informal idea : control points determine the shape of the curve. \textbf{Anchor points are on the curve} , \textbf{Handle points pull the curve}.

\textbf{Control polygon} joining the control points approximates the curve.

A spline may consist of several segments of curve.

    \begin{figure}[ht!]
  \centering
  \begin{subfigure}[b]{0.5\linewidth}
    \includegraphics[width=1\linewidth]{150.png}
  \end{subfigure}
\end{figure}

In math terms : s(t) = $\sum_i p_i B_i (t)$

The affine sum blends the contributions of the various control points. The enumeration of indices define the control polygon.

$p_i$ are the control points , $B_i$ is the base functions that weigh the influence of control points at each parameter value.

Basis functions characterize the type of spline.

\subsubsection{Bezier Curves}

Bases are polynomials of given degree n.

A Bezier segment of degree n has n+1 control points :

    \begin{figure}[ht!]
  \centering
  \begin{subfigure}[b]{0.6\linewidth}
    \includegraphics[width=1\linewidth]{151.png}
  \end{subfigure}
\end{figure}

\begin{itemize}
    \item n = 1 : straight line segment
    \item n = 2 : parabolic arc
    \item n = 3 : cubic arc
\end{itemize}

    \begin{figure}[ht!]
  \centering
  \begin{subfigure}[b]{0.4\linewidth}
    \includegraphics[width=1\linewidth]{152.png}
  \end{subfigure}
\end{figure}

For all n the curve : interpolates the endpoints of control polygon and is tangent to first and last segment of the control polygon at the end points


\textbf{Properties}

\begin{itemize}
    \item The $B_{i,n}$ are non-negative in [0,1]
    \item Partition of unity: $\sum_{i=0}^n B_{i,n} (t) = 1$
    \item Interpolation at endpoints: $B_{0,n} (0) = B_{n,n}(1) = 1$
    \item Symmetry : $B_{i,n} (t) = B_{n-1,n}(1-t)$
\end{itemize}


    \begin{figure}[ht!]
  \centering
  \begin{subfigure}[b]{0.6\linewidth}
    \includegraphics[width=1\linewidth]{153.png}
  \end{subfigure}
\end{figure}

   \begin{figure}[ht!]
  \centering
  \begin{subfigure}[b]{0.6\linewidth}
    \includegraphics[width=1\linewidth]{154.png}
  \end{subfigure}
\end{figure}

   \begin{figure}[ht!]
  \centering
  \begin{subfigure}[b]{0.6\linewidth}
    \includegraphics[width=1\linewidth]{155.png}
  \end{subfigure}
\end{figure}


\textbf{Convex hull property} : The curve is contained in the convex hull of its control points

\textbf{Variation diminishing property} : no straight line can have more intersection with the curve than with the control polygon.

    \begin{figure}[ht!]
  \centering
  \begin{subfigure}[b]{0.8\linewidth}
    \includegraphics[width=1\linewidth]{156.png}
  \end{subfigure}
\end{figure}

Disadvantages : 

\begin{itemize}
    \item Global support: moving each single control point modifies the whole curve and the change is not intuitive
    \item Numerical instability at high degree / large number of control points
    \item Low expressive power : conics such as circles cannot be represented
\end{itemize}

\subsection{de Casteljau Algorithm}

Bezier curves admit recursive definition : $b_i^0 = p_i$ , $b_i^r = (1-t)b_i^{r-1}(t) + tb_{i+1}^{r-1}(t)$

$s(t) = b_0^n(t)$

This definition immediately provides a geometric construction to evaluate the curve by means of repeated affine averages.

The de Casteljau algorithm computed at any value t splits the curve into two Bezier segments, providing their control polygons $b_0^0$, $b_0^1$, $b_0^2$, $b_0^3$ and  $b_0^3$, $b_1^2$, $b_2^1$, $b_0^3$

    \begin{figure}[ht!]
  \centering
  \begin{subfigure}[b]{0.5\linewidth}
    \includegraphics[width=1\linewidth]{157.png}
  \end{subfigure}
\end{figure}

Recursive evaluation at t = 1/2 provides a sequence of control polygons that collapses onto the curve.

\subsection{Bezier splines}

Bezier curves can be joined easily with $C^0$ and $C$

$C^0$: the last control point of a curve and the first control point of the next curve coincide

    \begin{figure}[ht!]
  \centering
  \begin{subfigure}[b]{0.4\linewidth}
    \includegraphics[width=1\linewidth]{158.png}
  \end{subfigure}
\end{figure}

$G^1$ : the last edge of a control polygon and the first edge of the next control polygon are collinear

$C^1$ : the last edge of a control polygon and the first

    \begin{figure}[ht!]
  \centering
  \begin{subfigure}[b]{0.4\linewidth}
    \includegraphics[width=1\linewidth]{159.png}
  \end{subfigure}
\end{figure}

$C^2$ continuity involves the mutual positions of the last three control points of a curve and the first three of the next.  This may cause a loss of local control.

\subsubsection{Can we do better than Bezier?}

Wish list :

\begin{itemize}
    \item Local support : each control point should influence only one small portion of curve
    \item Smooth splines : it should be easy to join curve segments smoothly
    \item Expressive power : use a more flexible set of bases
\end{itemize}

\subsection{B-splines}

Basis functions with compact support and flexible.

Defined upon a set of knots that subdivide an interval on R.

Knots control the amplitude and the shape of each basis function.

Basis function determine the influence of each control point.

Each segment of spline is determined by a subset of control points.

    \begin{figure}[ht!]
  \centering
  \begin{subfigure}[b]{0.8\linewidth}
    \includegraphics[width=1\linewidth]{160.png}
  \end{subfigure}
\end{figure}

\textbf{Example}

    \begin{figure}[ht!]
  \centering
  \begin{subfigure}[b]{0.8\linewidth}
    \includegraphics[width=1\linewidth]{161.png}
  \end{subfigure}
\end{figure}

\textbf{Properties}

\begin{itemize}
    \item There are n = m-p bases of degree p on m intervals
    \item All non-negative
    \item Each basis spans p+1 intervals
    \item Consecutive bases overlap
    \item Partition of unity in $[u_p,u_{m-p}]0$ : $\sum_{i=0}^{n-1} N_{i,p} (t) = 1$
\end{itemize}

B-spline curve :

    \begin{figure}[ht!]
  \centering
  \begin{subfigure}[b]{0.8\linewidth}
    \includegraphics[width=1\linewidth]{162.png}
  \end{subfigure}
\end{figure}

\textbf{Uniform B-splines :}

Completely defined by the n control points and the choice of p.

Have smoothness $C^{p-1}$, closed curves can be defined by using the knot interval in a periodic way.

\textbf{Non-uniform B-splines :} Knots can be moved to create non-uniform intervals. Non uniform intervals modify the shape of the basis function.

Different knots can be dragged to the same location to change smoothness of the curve

Each time we increase the multiplicity of a knot, we decrease smoothness by one.

A knot with multiplicity p forces the curve to interpolate one control point.

\textbf{Open-uniform B-splines} :

Set multiplicity p at the endpoints of the interval and uniform intervals inside.

We obtain a curve interpolating the endpoints while keeping the smoothness and control of B-splines.

    \begin{figure}[ht!]
  \centering
  \begin{subfigure}[b]{0.4\linewidth}
    \includegraphics[width=1\linewidth]{163.png}
  \end{subfigure}
\end{figure}

\subsection{de Boor Algorithm}

Generalization of de Casteljau algorithm for B-splines.

Exploits the recursive definition of B-spline bases:



    \begin{figure}[ht!]
  \centering
  \begin{subfigure}[b]{0.7\linewidth}
    \includegraphics[width=1\linewidth]{164.png}
  \end{subfigure}
\end{figure}

\subsection{Knot Insertion}

Purpose : represent the same curve with more knots to support more flexible editing.

Problem: Given a B-spline with m knots, find another B-spline with m+1 knots that describes the same curve.

Boehm algorithm provides a simple formula to compute the positions of the control points upon insertion of new node.

Oslo algorithms generalize to the simultaneous insertion of more nodes.


\subsection{Surfaces}

Idea : sweep a curve in space by dragging its control points along other curves.

    \begin{figure}[ht!]
  \centering
  \begin{subfigure}[b]{0.8\linewidth}
    \includegraphics[width=1\linewidth]{165.png}
  \end{subfigure}
\end{figure}

In math terms :

    \begin{figure}[ht!]
  \centering
  \begin{subfigure}[b]{0.8\linewidth}
    \includegraphics[width=1\linewidth]{166.png}
  \end{subfigure}
\end{figure}

This scheme can be applied to obtain surfaces from all piecewise polynomial schemes of curves

\subsubsection{Bezier surfaces}

    \begin{figure}[ht!]
  \centering
  \begin{subfigure}[b]{0.8\linewidth}
    \includegraphics[width=1\linewidth]{167.png}
  \end{subfigure}
\end{figure}

\subsubsection{B-spline surfaces}

    \begin{figure}[ht!]
  \centering
  \begin{subfigure}[b]{0.9\linewidth}
    \includegraphics[width=1\linewidth]{168.png}
  \end{subfigure}
\end{figure}


\vspace{60mm}

\section{Geodesics on Surfaces}

\subsection{Curve on a surface}

M surface.

p,q points on M.

$\gamma$ curve on M joining p to q:

    \begin{figure}[ht!]
  \centering
  \begin{subfigure}[b]{0.4\linewidth}
    \includegraphics[width=1\linewidth]{169.png}
  \end{subfigure}
\end{figure}


\begin{itemize}
    \item $\gamma [0,l_\gamma$ \textrightarrow{ M}
    \item $\gamma[0] = p$ , $\gamma[l_\gamma$]
    \item $l_\gamma $ of $\gamma$
\end{itemize}

\subsection{Geodesic distance on a surface}

Distance between A and B : length of the shortest curve joining A to B : d(A,B) = min $l_\gamma$

If d(A,B) = $l_{\bar \gamma}$ then $\bar \gamma$ is called a \textbf{shortest geodesic (path) between A and B}

The geodesic distance is unique but the shortest geodesic path is not always unique.

    \begin{figure}[ht!]
  \centering
  \begin{subfigure}[b]{0.4\linewidth}
    \includegraphics[width=1\linewidth]{170.png}
  \end{subfigure}
\end{figure}

\subsection{Cut Locus}

    \begin{figure}[ht!]
  \centering
  \begin{subfigure}[b]{0.4\linewidth}
    \includegraphics[width=1\linewidth]{173.png}
  \end{subfigure}
\end{figure}


\subsection{Geodesic Lines}

A curve $\gamma$ is a geodesic line if it bends with M but it does not bend on M.

    \begin{figure}[ht!]
  \centering
  \begin{subfigure}[b]{0.9\linewidth}
    \includegraphics[width=1\linewidth]{171.png}
  \end{subfigure}
\end{figure}

In some sense , a geodesic line is straight on M.

Shortest paths are geodesic lines.

Not all geodesic lines are shortest paths.

Geodesic lines are locally shortest.


A geodesic line is fully characterized with : its starting point p , its tangent direction at p and its length.

Exponential map: $exp_p : T_p M -> M$

It maps vectors from the tangent plane at a given point p into points on M.

The exp map of a vector t is the endpoint of a geodesic cast from p in the direction and for the length of t.

    \begin{figure}[ht!]
  \centering
  \begin{subfigure}[b]{0.5\linewidth}
    \includegraphics[width=1\linewidth]{172.png}
  \end{subfigure}
\end{figure}

\subsection{Normal and Convex sets}

A submanifold B contained in M is a \textbf{Normal neighborhood} of point x if for each t $\in$ B there exists a unique shortest ath on M from x to y.

A \textbf{totally normally set} if it is an open set that is a normal neighborhood of all its points.

\textbf{Strongly convex} if for each pair x and y of its points there exists a unique shortest path on M connecting x to y and it is contained in B

\subsection{Polyhedral setting}

On a polygonal mesh , geodesic lines are polylines.

One straight-line segment per face crossed.

The geodesic line is a straight line once the crossed strip of faces is unfolded to the plane.

   \begin{figure}[ht!]
  \centering
  \begin{subfigure}[b]{0.7\linewidth}
    \includegraphics[width=1\linewidth]{174.png}
  \end{subfigure}
\end{figure}


\begin{figure}[ht!]
  \centering
  \begin{subfigure}[b]{0.49\linewidth}
    \includegraphics[width=1\linewidth]{175.PNG}
  \end{subfigure}
     \begin{subfigure}[b]{0.49\textwidth}
         \centering
         \includegraphics[width=1\textwidth]{176.PNG}
     \end{subfigure}
\end{figure}

   \begin{figure}[ht!]
  \centering
  \begin{subfigure}[b]{0.8\linewidth}
    \includegraphics[width=1\linewidth]{177.png}
  \end{subfigure}
\end{figure}

\vspace{20mm}


\section{Computational Geodesics}

\subsection{Geodesic Tracing}

We assume that the starting point is a vertex $v_i$ (points on edges or inside triangles are easier to handle).

Tangent direction t is given on the tangent plane $T_i$M at $v_i$

Basic steps :

\begin{enumerate}
    \item Project t on M and find the triangle f crossed by such projection
    \item Propagate the triangle adjacent to f by crossed edge
    \item Repeat step 2 until reaching distance $|t| $
\end{enumerate}

Mapping the 1-ring to the tangent plane : 

\begin{itemize}
    \item Compute the total angle $\Theta_i$ about vertex $v_i$
    \item Rescale all incident angles by $2\pi/\Theta_i$
    \item Map one edge to the x axis and distribute the others by rescaled angles.
\end{itemize}

   \begin{figure}[ht!]
  \centering
  \begin{subfigure}[b]{0.2\linewidth}
    \includegraphics[width=1\linewidth]{178.png}
  \end{subfigure}
\end{figure}


\vspace{80mm}

Mapping the tangent direction to the mesh:



\begin{itemize}
    \item Find the triangle in tangent plane containing direction $\bar t$
    \item Rescale  angle by $\Theta_i/2pi$
    \item Map angle to the corresponding triangle of M
\end{itemize}

   \begin{figure}[ht!]
  \centering
  \begin{subfigure}[b]{0.2\linewidth}
    \includegraphics[width=1\linewidth]{179.png}
  \end{subfigure}
\end{figure}



Propagating the direction :

\begin{itemize}
    \item Find the outgoing edge of the direction in the current triangle
    \item Unfold the neighboring triangle on the same plane and propagate a straight line through it.
    \item Repeated unfolding gives a triangle strip
\end{itemize}

   \begin{figure}[ht!]
  \centering
  \begin{subfigure}[b]{0.6\linewidth}
    \includegraphics[width=1\linewidth]{180.png}
  \end{subfigure}
\end{figure}

What to do if the geodesic crosses a vertex?

\textbf{Straightest geodesic} : Extend with line that halves the total angle

\subsubsection{Single source Geodesic Distance}

Three classes of algorithms:

\begin{itemize}
    \item \textbf{Graph-based}: restrict possible routes to paths in a graph; compute distance function on the graph
    \item \textbf{PDE-based}: resolve a global PDE whose result approximates the distance function
    \item \textbf{Polyhedral wavefront propagation}: propagate a front through the mesh while building a data structure that supports computing the exact polyhedral distance from any point.
\end{itemize}

\vspace{20mm}

\subsection{Graph-Based Techniques}

\subsubsection{Shortest path on graphs}

   \begin{figure}[ht!]
  \centering
  \begin{subfigure}[b]{0.9\linewidth}
    \includegraphics[width=1\linewidth]{181.png}
  \end{subfigure}
\end{figure}

Assume all weights are non-negative. Let s $\in$ V , it is well defined the tree of shortest paths rooted at s , $G_s = (V_s,E_s)$ where $V_s$ is the set of vertices that can be reached from s and for each v in $V_s$ the simple path from s to v in $G_s$ is a shortest path in G.

\subsubsection{Optimal substructure of shortest paths}

   \begin{figure}[ht!]
  \centering
  \begin{subfigure}[b]{0.9\linewidth}
    \includegraphics[width=1\linewidth]{182.png}
  \end{subfigure}
\end{figure}

\subsubsection{Graph Relaxation}

For each node v let us define :

\begin{itemize}
    \item d[v] candidate length of shortest path , initialized at inf
    \item $\pi$[v] node preceding v in a candidate path , initialized empty
    \item Relaxation of an arc(u,v) : if d[v] $>$ d[u] + w(u,v) then d[v] = d[u] + w(w,v) , $\pi$(v) = ($\pi$(u), (u,v))
\end{itemize}

If we repeatedly relax all arcs of E the value of d[v] is always greater or equal than $\delta(s,v)$ and if d[v] reaches the value $\delta(s,v)$ then it becomes stable.

If $<s,...,u,v>$ is the shortest path between s and v and we have d[u] = $\delta(s,u)$ then after relaxing edge (u,v) we have d[v] = $\delta(x,v)$


\subsubsection{Dijkstra Expansion}

Input : graph G = (V,E) weight w , source s.

Output : graph G = (V,E) where for each v we have d[v] = $\delta(s,v)$ and $\pi$(v) points to his predecessor on the shortest path from s. 

Shortest paths can be extracted next in optimal time by backtracking from destination to source following $\pi(v)$

   \begin{figure}[ht!]
  \centering
  \begin{subfigure}[b]{0.7\linewidth}
    \includegraphics[width=1\linewidth]{183.png}
    \caption{Example of dijkstra}
  \end{subfigure}
\end{figure}

\subsubsection{SLF-LLL Heuristics}

The priority queue of Dijkstra warrants optimal worst-case time , but it is expensive to maintain in practice.

For planar graphs and graphs used in geodesic algorithms , an algorithm using a double ended queue achieves better practical performances, while being not optimal in the worst case :

\begin{itemize}
    \item \textbf{Small Label First (SLF)} : a new node is added either to the front or to the back of the queue , depending on its relative cost with respect to the first node of the queue.
    \item \textbf{Large Label Last (LLL)}: nodes from the front of the queue that have a cost higher than the average are moved to the back of the queue.
\end{itemize}

\subsubsection{Which graph to use?}

The graph restricts the possible sources to its nodes , and the possible paths to graph paths.

Trade-off : find a graph that achieves a good accuracy with a small number of nodes/arcs

In most cases , defining the distance field just at the vertices of a mesh is sufficient.


\subsubsection{Which graph to use?}

\textbf{Graph of edges}: V vertices of the mesh , E edges of the mesh. Easy and cheap but inaccurate.

   \begin{figure}[ht!]
  \centering
  \begin{subfigure}[b]{0.3\linewidth}
    \includegraphics[width=1\linewidth]{184.png}
  \end{subfigure}
\end{figure}

\textbf{Graph of adjancencies}: V centroids of triangles of the mesh, E dual edges = adjacencies between triangles of the mesh. Easy and cheep but inaccurate

   \begin{figure}[ht!]
  \centering
  \begin{subfigure}[b]{0.3\linewidth}
    \includegraphics[width=1\linewidth]{185.png}
  \end{subfigure}
\end{figure}

\vspace{40mm}

\textbf{Graph of primal and dual edges}: V vertices of the mesh , E edges of the mesh and dual edges. Each vertex is connected with its neighbors and with the vertices opposite to it in the triangles adjacent to its 1-ring

   \begin{figure}[ht!]
  \centering
  \begin{subfigure}[b]{0.3\linewidth}
    \includegraphics[width=1\linewidth]{187.png}
  \end{subfigure}
\end{figure}


\textbf{Graph with Steiner points on edges} : edges of the mesh are split by distributing Steiner points along them , within each triangle , arcs joining mutually visible points on its boundary are added. Easy but add extra nodes.

   \begin{figure}[ht!]
  \centering
  \begin{subfigure}[b]{0.3\linewidth}
    \includegraphics[width=1\linewidth]{186.png}
  \end{subfigure}
\end{figure}

\textbf{Discrete Geodesic Graph}: V vertices of the mesh, E subset of arcs of the total graph. Each arc is an exact shortest path between its endpoints and any exact path can be approximated with a path in the graph within a given tolerance $\epsilon$

   \begin{figure}[ht!]
  \centering
  \begin{subfigure}[b]{0.3\linewidth}
    \includegraphics[width=1\linewidth]{188.png}
  \end{subfigure}
\end{figure}

\subsection{Exact Polyhedral Methods}

Input : Mesh M and source s

Output : An explicit representation of the geodesic distance function $d_s$ : M \textrightarrow{ R} computable at any point p on M . An explicit representation of geodesics shortest paths $\delta(s,p)$

\subsubsection{Polygonal Wavefront Expansion}

Algorithm MPP and many subsequent improvements.

Basic Idea : expand a front of edges starting at the neighborhood of source s until covering the whole surface. Encode distance function $d_s$ for points on each edge e as a set of windows splitting e

\subsubsection{Geodesics through a triangle strip}

Assume s is a vertex.

Consider a triangle strip starting at s and flatten it to the plane.

Assume there exists at least one straight line from s contained inside the strip and crossing all its triangles (if not we are not interested in that strip).

The folding of such line to M is a shortest geodesic connecting s to its other endpoint.

  \begin{figure}[ht!]
  \centering
  \begin{subfigure}[b]{0.5\linewidth}
    \includegraphics[width=1\linewidth]{189.png}
  \end{subfigure}
\end{figure}

The set of all lines from s through the strip , and piercing the same edge e on the opposite end of it , form a wedge , encoded as window w on e :

\begin{itemize}
    \item parametric coordinates $b_0$ and $b_1$ of endpoints of w on e
    \item Distances $d_0$ and $d_1$ of endpoints of w from s
    \item Binary direction $\tau$ specifying the side of e on which s lies
\end{itemize}

  \begin{figure}[ht!]
  \centering
  \begin{subfigure}[b]{0.5\linewidth}
    \includegraphics[width=1\linewidth]{190.png}
  \end{subfigure}
\end{figure}

\vspace{20mm}

\subsubsection{Distance function in a window}

  \begin{figure}[ht!]
  \centering
  \begin{subfigure}[b]{0.8\linewidth}
    \includegraphics[width=1\linewidth]{191.png}
  \end{subfigure}
\end{figure}

\subsubsection{General Case}

Not all points can be reached from straight lines through strips.

What happen at saddle vertices? The unfolding of the 1-ring overlaps , points in the red wedge cannot be reached by straight lines from s , so we need to turn about p to reach the red window w. Line l is a mandatory route to all points of w.

  \begin{figure}[ht!]
  \centering
  \begin{subfigure}[b]{0.4\linewidth}
    \includegraphics[width=1\linewidth]{192.png}
  \end{subfigure}
\end{figure}

Then we decompose the paths to a generic window w into a \textbf{funnel} consisting of two parts : a polygonal path of length $\sigma$ from source s to pseude-source p and a wedge from p to w.

\vspace{30mm}

The distance function for all x in w is encoded as : $(b_0,b_1,d_0,d_1,\sigma,\tau)$

  \begin{figure}[ht!]
  \centering
  \begin{subfigure}[b]{0.5\linewidth}
    \includegraphics[width=1\linewidth]{193.png}
  \end{subfigure}
\end{figure}

\subsubsection{Window Propagation}

Given a window w on an edge $e_1$ propagate its distance field across an adjacent triangle t to define new potential windows on the opposing edges $e_2,e_3$.

  \begin{figure}[ht!]
  \centering
  \begin{subfigure}[b]{0.3\linewidth}
    \includegraphics[width=1\linewidth]{194.png}
  \end{subfigure}
\end{figure}

To define the distance field over w' : extend the rays from the pseudo-source s through the endpoints of w and intersect them wit the new edge , to obtain the new interval $[b'_0,b'_1]$. Compute the new distances $d'_0,d'_1$ from pseudo-source s. Pseudo-source distace $\sigma' = \sigma$ is unchanged. Direction $\tau$ is assigned to point inside triangle (refers to the image above).

Special case : if one of the endpoints of w is a saddle vertex $p_0$ : Consider the other edge $\bar e$ of face t incident at $p_0$. If $\bar e$ outside the funnel , the generate the two new red windows in the figure, having $p_0$ as pseudo-source. Distance $\sigma$ must be updated $\sigma = \sigma + d_0$

\subsubsection{Continuous Dijkstra propagation}

Maintain a priority queue Q of windows. Prioriry is minimum distance from source.

Initialization of Q : 

\begin{itemize}
    \item If is inside triangle t, then compute one window per edge of t and insert them in the queue.
    \item If s is on an edge e , then compute one window per edge of the two faces incident at e and insert them in the queue. Also compute two windows on e.
    \item If s is a vertex , then every triangle t in the star of s compute one window per edge of t opposite to s and insert them in the queue. Also compute one window for each edge e incident at s.
\end{itemize}

Loop while Q becomes empty :

\begin{itemize}
    \item Pop a window w from Q
    \item Propagate window w
    \item Insert each non-empty window generated by propagation into Q
\end{itemize}

Propagation may:

\begin{itemize}
    \item Add a new window
    \item Modify existing windows
    \item Delete existing windows
\end{itemize}

\subsubsection{Geodesic path Construction}

Once all edges are covered by windows representing distance, it is easy to trace a shortest path from any surface point p back to the source.

For a point p inside a face r: minimize $|p-p'| + d_s(p') $ for all p' on the edges of t. Jump to the window containing the point that minimizes that distance.

For a point p on a window: Find the adjacent triangle t according to direction $\tau$. Reconstruct the position of the pseudo-source s in the plane of t and intersect the line to s with the other two edges of t. Jump to the intersection point , which is on new window , and repeat.
When reaching a pseudo-source s : explore the windows incident at s until a window with a pseudo-source different from s is found.

Repeat until reaching the source.

\subsection{PDE-Based Methods}

\subsubsection{The Eikonal Equation}

The SSGDD problem can be formalized as a PDE:

$|\bigtriangledown \phi|^2 = 1$ , $\phi(s) = 0$

The change in distance per unit distance along the direction of greatest increase should be 1.

Geodesic lines are integral curves of the gradient of the distance field.

\subsubsection{The Fast Marching Method}

Following a Dijkstra-like traversal of the mesh, starting at the source.

Distances are not updated according to paths along edges, but by solving for the linear function that satisfies the eikonal equation.

If the values $\phi_i,\phi_j$ at two corners are known , find a value $\phi_k$ so that the slope $|\bigtriangledown \phi|$ of a triangle passing through all three values equals +1.

 \begin{figure}[ht!]
  \centering
  \begin{subfigure}[b]{0.3\linewidth}
    \includegraphics[width=1\linewidth]{195.png}
  \end{subfigure}
\end{figure}

\vspace{20mm}

\subsubsection{The Varadhan Formula}

 \begin{figure}[ht!]
  \centering
  \begin{subfigure}[b]{0.8\linewidth}
    \includegraphics[width=1\linewidth]{196.png}
  \end{subfigure}
\end{figure}

The heat equation is linear and can be resolved easily.

Heat decays exponentially with distance, hence the norm of the result becomes soon unreliable : the smaller t the faster the decay , the bigger t the worse approximation of Varadhan formula

\subsubsection{The Heat Method}

 \begin{figure}[ht!]
  \centering
  \begin{subfigure}[b]{0.8\linewidth}
    \includegraphics[width=1\linewidth]{197.png}
  \end{subfigure}
\end{figure}

\subsection{Local Methods for PPGP}

\subsubsection{Point to Point Geodesic Path}

\begin{enumerate}
    \item Start with a triangle strip joining p to q, usually found with a graph-based method
    \item Unfold strip to the plane, one triangle at a time by using intersecting circles
    \item Find shortest path within the strip , Funnel algorithm
    \item Straighten the strip to remove turns, Straightening the strip
    \item Repeat 3 and 4 until convergence
\end{enumerate}

 \begin{figure}[ht!]
  \centering
  \begin{subfigure}[b]{0.2\linewidth}
    \includegraphics[width=1\linewidth]{198.png}
  \end{subfigure}
\end{figure}

\subsubsection{Shortest path in a triangle strip}

Input: triangle strip in the plane, containing p and q in its end triangles , respectively.

Output: Shortest polyline connecting p to q inside the triangle strip.

The path can turn only at reflex vertices

 \begin{figure}[ht!]
  \centering
  \begin{subfigure}[b]{0.3\linewidth}
    \includegraphics[width=1\linewidth]{199.png}
  \end{subfigure}
\end{figure}

Initialize funnel : connect p to the endpoints of first transversal edge of the strip; the mouth of the funnel spans all points that can be reached from p

 \begin{figure}[ht!]
  \centering
  \begin{subfigure}[b]{0.3\linewidth}
    \includegraphics[width=1\linewidth]{200.png}
  \end{subfigure}
\end{figure}

\vspace{30mm}

Expand funnel : move one side of the funnel to the next transversal edge; the sides of the funnel may become concave chains when they overcome reflex edges.

 \begin{figure}[ht!]
  \centering
  \begin{subfigure}[b]{0.3\linewidth}
    \includegraphics[width=1\linewidth]{201.png}
  \end{subfigure}
\end{figure}

While advancing , the sides of funnel they may partially collapse to form the beak.

Collapses happen when updating one side overcomes the other side.

\begin{figure}[ht!]
  \centering
  \begin{subfigure}[b]{0.29\linewidth}
    \includegraphics[width=1\linewidth]{202.PNG}
  \end{subfigure}
     \begin{subfigure}[b]{0.39\textwidth}
         \centering
         \includegraphics[width=1\textwidth]{203.PNG}
         \caption{The funnel stops advancing when it reaches the last triangle}
     \end{subfigure}
\end{figure}

\subsubsection{Straightening the strip}

\begin{figure}[ht!]
  \centering
  \begin{subfigure}[b]{0.8\linewidth}
    \includegraphics[width=1\linewidth]{204.png}
  \end{subfigure}
\end{figure}

\vspace{40mm}

\section{Vector Graphics}

\subsection{Vector Graphics}

Specify graphics drawings as composed of \textbf{vector primitives}.

A vector primitive is defined with:

\begin{itemize}
    \item A set of control points in a reference system
    \item Possible additional parameters
    \item Math to uniquely identify the primitive from the control points and the parameters.
\end{itemize}

Examples

A straight line segment is defined by two control points p and q and parametric equation p+$\alpha$(p-q).

A circle is defined by point c radius r and equation $|p-c|^2 = r^2$


\subsubsection{Vector Graphics on a mesh}

Can we extend vector graphics to a manifold domain, like a surface mesh?

We should redefine everything under the geodesic metric: straight line can be replaced with geodesic lines. What about angles? Ellipses? 


\subsubsection{Referencing points on a mesh}

We can assign mesh coordinates to all points of a mesh M.

Point p has coordinates ($t_p,a_p,b_p$) where :

\begin{itemize}
    \item $t_p$ is the triangle of M containing p
    \item $a_p.b_p$, are the barycentric coordinates of p in t. A tuple of coordinates uniquely identifies a point of M
\end{itemize}

\vspace{60mm}

\subsection{Simple Vector Primitives}
\subsubsection{Geodesic line segment}

Defined by a pair of points p,q.

We assume this is the shortest geodesic joining them and it is unique. 

\begin{figure}[ht!]
  \centering
  \begin{subfigure}[b]{0.3\linewidth}
    \includegraphics[width=1\linewidth]{205.png}
  \end{subfigure}
\end{figure}

\subsubsection{Geodesic Circle}

Defined by center c and radius r : SSGD from c gives distance field $d_c$ at all vertices of M.

\begin{figure}[ht!]
  \centering
  \begin{subfigure}[b]{0.3\linewidth}
    \includegraphics[width=1\linewidth]{206.png}
  \end{subfigure}
\end{figure}

It is defined by three points $p_1,p_2,p_3$  : we need to compute the distance fields from these points with SSGD , find the point c where the three distance fields have the same value and the corresponding value r and find the circle centered at c of radius r with previous method.

\subsubsection{Triangle}

Defined by its vertices $p_1,p_2,p_3$ :

\begin{enumerate}
    \item Connect the vertices with shortest geodesics with PPGP
    \item Find the interior of triangle by flooding the surface from such geodesics.
\end{enumerate}

\subsubsection{Isosceles triangle}

Defined by basis and height or defined by basis and angle at basis or defined by basis and length of diagonal edges

\begin{figure}[ht!]
  \centering
  \begin{subfigure}[b]{0.3\linewidth}
    \includegraphics[width=1\linewidth]{207.png}
  \end{subfigure}
\end{figure}

\subsubsection{Equilateral triangle}

Defined by its basis. Different constructions are possible , but only one of the following properties can be guaranteed : all edges have equal length or all angles are equal.

\subsubsection{Rectangle}

Several possible definitions.

If edges are geodesics , the basic properties of Euclidean triangles cannot be guaranteed

\subsection{Bezier Curves}

Directly translating the Bernstein or De Casteljau formulas does not work in general case.

A recursive application of the De Casteljau construction defines a subdivision scheme that converges to a $C^1$ curve

\textbf{Example for a cubic curve}

\begin{figure}[ht!]
  \centering
  \begin{subfigure}[b]{0.3\linewidth}
    \includegraphics[width=1\linewidth]{208.png}
  \end{subfigure}
\end{figure}

Given the initial control polygon (blue):

\begin{enumerate}
    \item Build the De Casteljau construction for evaluating the midpoint of the curve ( the control polygon is substituted with a chain of two control polygons joining at the midpoint
    \item Apply Step 1 recursively to each sub-polygon
    \item Stop when the control chain is straight enough
\end{enumerate}


Each recursion step requires:

\begin{enumerate}
    \item finding the midpoint of three existing geodesics $b_0^1,b_2^1,b_1^1$
    \item Computing geodesics $b_0^1,b_1^1$ and $b_2^1,b_1^1$ and their midpoints $b_0^1,b_2^1$
    \item Computing geodesics $b_0^2,b_1^2$ and its midpoint b
    \item Halving five existing geodesics
\end{enumerate}

\begin{figure}[ht!]
  \centering
  \begin{subfigure}[b]{0.8\linewidth}
    \includegraphics[width=1\linewidth]{209.png}
  \end{subfigure}
\end{figure}

\end{document}
